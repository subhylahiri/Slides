% -*- TeX -*- -*- UK -*- -*- BMR -*- -*- PST -*-
% ----------------------------------------------------------------
% Beamer  Poster presentation ************************************************
%
% Subhaneil Lahiri's template
%
% To compile:
%   Ctrl-Shift-P
%
% **** -----------------------------------------------------------
\documentclass[final,hyperref={pdfpagelabels=false}]{beamer}
%\documentclass{beamer}
\usetheme{Subhy}
  \usepackage{times}
%  \usepackage{amsmath,amsthm, amssymb, latexsym}
%  \boldmath
\usepackage[orientation=Landscape,size=a0,scale=1.4,debug]{beamerposter}

%---------Packages-------------------------------------------------------

% For double screen:
%\usepackage{pgfpages}
%\setbeameroption{show notes on second screen=right}
%
% For finding documentation:
%\usepackage{ams}
%\usepackage[centertags]{amsmath}
%\usepackage{amssymb}
%\usepackage{xcolor}
%\usepackage{pgf}
%\usepackage{graphicx}
%
%\usepackage{ifpdf}
\ifpdf
\else
\DeclareGraphicsRule{.png}{eps}{.bb}{}
\fi
\usepackage{epstopdf}
\epstopdfsetup{update,suffix=-generated}
%---------Colours---------------------------------------------------------

% \newrgbcolor{LemonChiffon}{1. 0.98 0.8}
% \newrgbcolor{myellow}{.9 .8 .1}
% \newrgbcolor{myblue}{.2 .36 .77}
% \newrgbcolor{orange}{0.8 0.7 0.2}
% \newrgbcolor{myred}{0.95 0.0 0.0}
\definecolor{darkgrey}{rgb}{.5 .5 .5}
\definecolor{darkblue}{rgb}{0.27843137 0.27058824 0.5372549}
\definecolor{darkred}{rgb}{0.5372549 0.27843137 0.27058824}

%---------Commands-------------------------------------------------------

\newcommand{\rref}[1]{\hfill \small{\color{darkgrey} [#1]}}
\newcommand{\rrref}[1]{ {\color{darkgrey} #1}}
\newcommand{\citerr}[1]{\hfill {\footnotesize{\color{darkgrey}\cite{#1}}}}

\input{mydefs.tex}
\input{slidesymb.tex}

\DeclareMathOperator{\SNR}{SNR}
\DeclareMathOperator{\snr}{SNR}
\newcommand{\net}{molecular network}
\newcommand{\Net}{Molecular network}

%%%%%%%%%%%%%%%%%%%%%%%%%%%%%%%%%%%%%%%%%%%%%%%%%%%%%%%%%%%%%%%%%%%%%%%%%%%

%---------Title-----------------------------------------------------------

\title{Learning and memory with complex synapses}
%
%\subtitle{\small{based on \texttt{arXiv: [hep-th]} with }}
%
\author{Subhaneil Lahiri and Surya Ganguli}
%
\institute[Stanford]{%
Department of Applied Physics, Stanford University, Stanford CA
}
\date[6/27/12]{June 27, 2012}%

%%%%%%%%%%%%%%%%%%%%%%%%%%%%%%%%%%%%%%%%%%%%%%%%%%%%%%%%%%%%%%%%%%%%%%%%%%%

%---------Setup--------------------------------------------------------

\begin{document}

\begin{frame}{}

\begin{columns}[t]

%%%%%%%%%%%%%%%%%%%%%%%%%%%%%%%%%%%%%%%%%%%%%%%%%%%%%%%%%%%%%%%%%%%%%%%%%%%
%-------------Beginning--------------------------------------------------------

%-------------Column--------------------------------------------------------
\begin{column}{0.32\linewidth}

%-------------Box--------------------------------------------------------
\section{Background}
\begin{block}{Storage capacity of Hopfield networks}
%
 \begin{itemize}
   \item Hopfield capacity $\propto N$ requires unbounded synaptic strengths.
   \item Bounded synapses $\implies$ capacity $\propto\log N$.
   \item Can be ameliorated by using complex synapses
 \end{itemize}
%
\end{block}


\end{column}

%-------------Column--------------------------------------------------------
\begin{column}{0.32\linewidth}


%-------------Box--------------------------------------------------------

\begin{block}{Area bound}
%
 We can show that the area under the SNR curve is bounded:
 %
 \begin{equation*}
   A \leq \sqrt{N}(n-1).
 \end{equation*}
 %
 This leads to a bound on the lifetime of a memory:
 %
 \begin{equation*}
   \SNR(\text{lifetime})=1
   \qquad
   A \geq \text{lifetime}.
 \end{equation*}
 %
 This is saturated by a \net with the multistate topology.
%
\end{block}

%-------------Box--------------------------------------------------------

\begin{block}{Ordering the states}
%
 Let 
%
\end{block}







\end{column}

%-------------Column--------------------------------------------------------
\begin{column}{0.32\linewidth}




%-------------Box--------------------------------------------------------

\begin{block}{References}
%
 {\small
 \bibliographystyle{unsrt_slides}
 \bibliography{neuro,maths}
 }
%
\end{block}


\end{column}




%%%%%%%%%%%%%%%%%%%%%%%%%%%%%%%%%%%%%%%%%%%%%%%%%%%%%%%%%%%%%%%%%%%%%%%%%%%
%-----End----------------------------------------------------------------

\end{columns}

\end{frame}

\end{document}
