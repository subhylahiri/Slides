% -*- TeX -*- -*- UK -*-

\section{Introduction}\label{sec:intro}

A particularly interesting entry in the dictionary between gauge
theory and gravity links deconfined or `gluon plasma' phase of Yang
Mills theory to black branes and black holes in gravity. In this
paper we study aspects of this connection in the context of specific
examples. In most of this paper we study  $d=4$, $SU(N)$, $\CN=4$
Yang-Mills at 't~Hooft coupling $g^2_{YM}N=\lambda$, compactified on
a Scherk-Schwarz $S^1$ (the remaining $2+1$ dimensions are non
compact). The low energy dynamics of this theory is that of a $2+1$
dimensional Yang-Mills system that undergoes deconfining phase
transition at a finite temperature \cite{Witten:1998zw}. At large
$N$ and strong 't~Hooft coupling this system admits supergravity
dual description; the low temperature confining phase is dual to a
gas of IIB supergravitons on the so called AdS soliton background
\cite{Witten:1998zw}
%
\begin{equation}\label{deconfmet:eq}
  \dr s^2 = L^2 \apm \left( \e^{2u}
       \left( -\dr t^2 + T_{2 \pi}(u)\, \dr\theta^2 + \dr w_i^2 \right)
       + \frac{1}{T_{2 \pi}(u)}\, \dr u^2 \right),
\end{equation}
%
where $i= 1,\cdots, 2$, $\theta\sim\theta+2\pi$, $L^2=
\sqrt{4\pi\lambda}$ and\footnote{Notice that, at large $u$,
$T_x(u)\simeq 1$, so \eqref{deconfmet:eq} reduces to AdS$_{d+2}$ in
Poincar\'e-patch coordinates, with $u$ as the radial scale
coordinate, and with one of the spatial boundary coordinates,
$\theta$, compactified on a circle (the remaining boundary
coordinates, $w_i$ and $t$, remain non-compact).}
%
\begin{equation}\label{AdSSchII:eq}
 T_{x}(u) = 1 - \left( \frac{x}{\pi} \,\e^u \right)^{-4}.
\end{equation}
%


The high temperature phase of the same system (at temperature
$\tloc=1/\beta$) is dual to the the black brane
%
\begin{equation} \label{AdSSchIb:eq}
  \dr s^2 = L^2 \apm \left( \e^{2 u}
     \left( -T_{\beta}(u)\, \dr t^2 + \dr\theta^2 + \dr w_i^2 \right)
     + \frac{1}{T_{\beta}(u)}\, \dr u^2 \right) .
\end{equation}
%
The thermodynamics of the high temperature phase are determined in
the bulk description by the usual constitutive equations of black
brane thermodynamics \cite{Aharony:2005bm}
%
\begin{equation}\label{equationofstate:eq}
P=-f=\frac{N^2} {2^{130} \pi^3 \tc}\left( (2 \pi \tloc)^4-(2 \pi
\tc)^4 \right).
\end{equation}
%
For $\tloc>\tc$ this free energy is negative, and so (in the large
$N$ limit) is smaller than the $\CO(1)$ free energy of the
`confined' gas of gravitons. Consequently, the system undergoes a
deconfinement phase transition at temperature $\tc$.\footnote{$\tc =
1/2 \pi$ in the dimensionless units of \eqref{AdSSchIb:eq}}

Just as the mean equilibrium properties of the deconfined phase are
well described by the equations of thermodynamics, the statistically
averaged near-equilibrium dynamics of this phase is governed by the
equations of fluid dynamics - the relativistic generalisation of the
Navier-Stokes equations. These equations accurately describe the
time evolution of fluid configurations whose space time derivatives
are all small in units of the mean free path, which is of the same
order as the mass gap of the theory \cite{Son:2007vk,
Aharony:2005bm}. The same equations, augmented by appropriate
surface terms, may also be used to study the dynamics of large lumps
of plasma localised in the gauge theory vacuum.

The properties of the surface that separates the plasma from the
vacuum, may be studied in the context of the simplest plasma profile
with a surface; a configuration in which half of space, $x<0$, is
filled with the plasma. The surface at $x=0$ is a domain wall that
separates the plasma from the vacuum. The net force on this domain
wall vanishes (and so the system is in equilibrium)  when the plasma
that fills $x<0$ has vanishing pressure, i.e. at $\tloc=\tc$ in the
large $N$ limit. The bulk gravity dual of this solution was
constructed numerically in \cite{Aharony:2005bm}; this configuration
interpolates between the black brane at $\tloc=\tc$ for $x<0$ and
the vacuum at $x>0$, via a domain wall. The thickness and surface
tension of this domain wall may be read off from this gravitational
solutions, and  were estimated, in \cite{Aharony:2005bm} at
approximately $6\times \frac{1}{2\pi \tc}$ and $\sigma=2.0 \times
\frac{\pi N^2 \tc^2}{128}$ .

More generally, one would expect a finite lump of plasma that
evolves according to the relativistic Navier-Stokes equations map in
the bulk to a `black hole' that evolves according to the Einstein
equations. Provided all length scales in the plasma solution are
small compared to the gauge theory mass gap (which is of the same
order as the domain wall thickness), the dual bulk solution is well
approximated by a superposition of patches of the black brane
solution (with temperature varying across the patches) in the bulk
and patches of the domain wall solution described in the previous
paragraph. It follows (at least for stationary solutions) that the 3
dimensional black hole horizon topology (at any given time) is given
by an $S^1$ (physically this is the $\theta$ circle) fibred over the
two dimensional fluid configuration at the same time, subject to the
condition that the $S^1$ contracts at all fluid boundaries.
Consequently, fluid configurations with different topologies yield
bulk dual black hole configurations with distinct horizon
topologies. We will return to this point below.

This paper is devoted to a detailed study of certain `stationary'
configurations of the plasma fluid; i.e. time independent, steady
state solutions to the relativistic Navier-Stokes equations. The
simplest configurations of this sort was studied already in
\cite{Aharony:2005bm}; the plasmaball is a static, spherically
symmetric lump of fluid at constant local pressure $P$ with $P=
\sigma /R$ where $R$ is the radius of the lump and $\sigma$ its
surface tension. In this paper we study the more intricate spinning
lumps of stationary fluid. These lumps carry angular momentum in
addition to their mass.

It turns out that the relativistic Navier-Stokes equations admit two
distinct classes of solutions of these sort. The first class of
solution is a simple deformation of the static plasmaball; it is
given by plasmaballs that spin at a constant angular velocity. The
centripetal force needed to keep the configuration rotating in this
solution is provided by a pressure gradient. The local plasma
pressure (and hence local temperature and density) decreases from
the edge (where it is a positive number set by the radius, surface
tension and rotation speed) to the centre. As large enough angular
velocity the pressure goes sufficiently negative in the core of the
solution to allow for a second kind of solution of these equations;
an annulus of plasma fluid rotating at constant angular velocity
$\omega$. The local plasma pressure is positive on the outer surface
and negative at the inner surface; the numerical value of the
pressure in each case precisely balances the surface tensions at
these boundaries.

We now describe the moduli space of spinning plasmaball and plasma
ring solutions in a little more detail. In
fig.\ref{exist_intro:fig}(a) we have plotted the energy-angular
momentum plane, which we have divided up into 4 regions. In region
$\tC$ (low angular momentum at fixed energy) the only rigidly
rotating solution to the equations of fluid dynamics is the rotating
plasmaball. At higher angular momentum (region $\tB$) in addition to
the rotating plasmaball there exist two new annulus type solutions
which we call large and small ring solutions. As their names makes
clear, the solutions are distinguished by their size; the large ring
has a larger outer radius than the small one. On further raising
angular momentum (region $\tA$), the small ring and the ball cease
to exist; in this region the large ring is the only solution.
Finally, at still larger angular momentum (region $\tOr$) there
exist no solutions.

\begin{figure}
%
 \begin{center}
  \small{$\tOr$ - no solutions, $\tA$ - large ring, $\tB$ - large ring, small ring
  and ball, $\tC$ - ball.}\\
%  \small{Green - ball, Red - large ring, Blue - small ring.} \\
   \small{(a)}%\includegraphics[width=4cm]{allreg.eps}
   \input{allreg.tpx}
   \hspace{2.5cm}
   \small{(b)}%\includegraphics[width=6cm]{ent.eps}
   \input{ent.tpx}
 \caption{(a) Regions where ball and ring solutions
exist, (b) their entropy as a function of angular momentum at fixed
energy.}\label{exist_intro:fig}
 \end{center}
\end{figure}

In fig.\ref{exist_intro:fig}(b) we have plotted the entropy of the
three different kinds of solutions as a function of their angular
momentum at a particular fixed energy. At angular momenta for which
all three solutions coexist (region $\tB$) the entropy of the small
ring is always smaller than the entropy of either the large ring or
the black hole. Upon raising the angular momentum, the solution with
dominant entropy switches from being the ball to the large ring; the
first order transition between these solutions occurs at an angular
momentum that lies on a `phase transition line' in the bulk of
region $\tB$. This picture suggests - and we conjecture - that the
ball and the large ring are locally stable with respect to
axisymmetric fluctuations, while the small ring is locally unstable
to such fluctuations. \footnote{It is possible that the large ring
exhibits Rayleigh-Plateau type instabilities that break rotational
invariance; such modes would map to Gregory-Laflamme type
instabilities of the bulk solution (see also \cite{Cardoso:2006ks}).
We thank T. Wiseman for suggesting this possibility.} In
\S\S\ref{sec:turn} we perform a `turning point' analysis of our
solutions, to find some evidence for this guess.

Let us now turn to the bulk dual interpretation of our solutions.
The fluid for the spinning  plasmaball is topologically a disk;
consequently the horizon topology for the dual bulk solution - the
$S^1$ fibration over this disk - yields an $S^3$. The bulk dual of
the spinning plasmaball is simply a rotating 5 dimensional black
hole. On the other hand the fluid configuration of the plasmaring
has the topology of $S^1\times$ interval; the $S^1$ fibration over
this configuration yields $S^1 \times S^2$; i.e. a five dimensional
black ring! Notice that in addition to the isometry along the $S^1$, 
these ring solutions all have a isometry on the $S^2$ corresponding to translations 
along the Scherk-Schwarz circle. This additional isometry, that does not appear 
to be required by symmetry considerations, appears to be a feature of all known black ring solutions in flat space as well.

Using the gauge theory / gravity duality, the quantitative versions
of the fig.\ref{exist_intro:fig} give precise quantitative
predictions for the existence, thermodynamic properties and
stability of sufficiently big  black holes and black rings in
Scherk-Schwarz compactified AdS$_5$ spaces. While these
gravitational solutions have not yet been constructed, their
analogues in flat 5 dimensional space are known, and have been well
studied. The general qualitative features (and some quantitative
features) of fig.\ref{exist_intro:fig} are in remarkably good
agreement with the analogous plots for black holes and black rings
in flat 5 dimensional space (see \S\S\ref{sec:existcomp} for a
detailed discussion).

The constructions we have described above admit simple
generalisations to plasma solutions dual to black holes and black
rings in Scherk-Schwarz compactified AdS$_6$ space.\footnote{Note
that the spinning plasmaring has no analogue in 1+1 dimensional
fluid dynamics, for the excellent reason that there is no spin. This
tallies with the fact that there are no black rings in four
dimensions (at least in flat space).} As the qualitative nature of
the moduli space of black hole like solutions in six dimensional
gravity is poorly understood, this study is of interest. The
boundary duals of these objects, in the long wavelength limit, are
stationary solutions to the equations of fluid dynamics of a 4
dimensional field theory. In \S\ref{sec:4dim} we construct such
solutions. It turns out that these solutions occur in two
qualitatively distinct classes. The simplest solutions are simply
spinning balls of plasma; the fact that these balls spin causes them
to flatten out near the `poles'. As these balls are spun up, their
profile begins to `dip' near the poles (see fig.\ref{4d:fig}). As
these balls are further spun up, they pinch off at the centre and
turn into doughnut shaped rings (see fig.\ref{4d:fig}).

\begin{figure}
%
\begin{center}
  \input{obs.tpx}
  \input{pbs.tpx}
  \input{rs.tpx}
  \caption{Spinning ball and ring solutions.}\label{4d:fig}
\end{center}
%
\end{figure}

As in the 3 dimensional case, the horizon topology of the black
objects dual to the rotating plasmaballs and plasmarings described
above, is obtained by fibering the fluid configuration with an $S^1$
that shrinks to zero at the fluid edges. This procedure yields a
horizon topology $S^4$ for the dual to the rotating plasmaball, and
topology $S^3\times S^1$ for the dual to the plasmaring. As
plasmaball and plasmaring configurations appear to exhaust the set
of stationary fluid solutions to the equations of fluid dynamics, it
follows that arbitrarily large stationary black objects in
Scherk-Schwarz compactified AdS$_6$ all have one of these two
horizon topologies. $S^2 \times S^2$ is an example of another
topology one could have imagined for black objects in this space;
these would have been dual to hollow shells of rotating fluid;
however, there are no such stationary solutions to the equations of
fluid dynamics.

The analysis of 4 dimensional fluid configurations, described above,
demonstrates the power of the fluid dynamical method. In simple
contexts, the Navier-Stokes equations are much easier to solve than
the full set of Einstein's equations, and rather easily reveal
interesting and nontrivial information. It would be interesting to
extend our analysis of fluid dynamical models in various directions
to obtain information about the moduli space and stability of
classes of black solutions in AdS spaces. An obvious extension would
be to move to higher dimensions. As a first step in this direction,
we have obtained and partially solved the fluid flow equations in 5
dimensional spaces. A complete analysis of these equations would
yield the spectrum of black holes in Scherk-Schwarz compactified
AdS$_7$ spaces, in terms of the fluid dynamics of the deconfined
phase of the M5 brane theory on a Scherk-Schwarz circle.

Finally, we should point out that there has been a long history
within the General Relativity literature of treating black hole
horizons as surfaces associated with fluids. In one of the most
recent discussion within this framework, the authors of
\cite{Cardoso:2007ka} have modelled spinning black holes in $d+1$
dimensional (flat space) gravity by $d+1$ dimensional lumps of
incompressible fluid; here the fluid surface represents the black
hole horizon. Within this framework the 4+1 dimensional black ring,
for instance, is modelled by a 4+1 dimensional stationary fluid lump
of topology $B^3\times S^1$ \cite{Cardoso:2006sj}. This description
is rather different from the AdS/CFT induced description of black
rings in Scherk Schwarz compactified AdS$_5$ as a 2+1 dimensional
annulus of fluid. It would be interesting to better understand the
interconnections between these approaches.
