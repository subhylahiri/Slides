% -*- TeX -*- -*- UK -*-

\section{Five dimensional plasmarings}\label{sec:5dim}

In the bulk of this paper we  have presented an analysis of
stationary fluid configurations of the 3 and 4 dimensional fluid
flows. The analysis of analogous configurations in one higher
dimension has an interesting new element. The rotation group in 4
spatial dimensions, $SO(4)$, has rank 2. Consequently a rotating
lump of fluid in 5 dimensions will be characterised by 3 rather than
two conserved charges (two angular momenta plus energy). When one of
the two angular momenta is set to zero, it seems likely that the set
of stationary solutions will be similar to those of the four
dimensional fluid; in this limit we expect ball and ring
configurations whose dual bulk horizon topologies are $S^5$ and
$S^4\times S^1$ respectively. However turning on the second angular
momentum on the ring solution could centrifugally repel the fluid
away from the second rotational axis, leading to a fluid
configuration with dual bulk horizon topology $S^1\times S^1\times
S^3$. Such configurations have not yet been discovered in gravity,
and it would be exciting to either construct them in fluid
mechanics, or to rule out their existence.

In this appendix we set up and partially solve the equations of
stationary fluid flow in 5 dimensions. While the stationary
equations of fluid dynamics are trivial to solve in the bulk in
every dimension, boundary conditions are harder to impose in higher
dimensions. In the particular case of 5 dimensions, the imposition
of these boundary conditions requires the solution of a 2nd order
ordinary differential equation, that we have not (yet?) been able to
solve. It may be that a full study of this case would require
careful numerical analysis, which we leave to future work. In the
rest of this appendix we simply set up the relevant equations, and
comment on the dual bulk interpretations of various possible
solutions.

Consider a fluid propagating in flat 5 dimensional space
%
\begin{equation*}
  \dr s^2 = -\dr t^2 + \dr r_1^2 + r_1^2 \dr \phi_1^2
              + \dr r_2^2 + r_2^2 \dr \phi_2^2\,.
\end{equation*}
%
Consider a fluid flow with velocities given by
$u^\mu=\gamma(1,0,\omega_1,0,\omega_2)$, where
$\gamma=(1-v_1^2-v_2^2)^{-1/2}$, $v_1=\omega_1r_1$ and
$v_2=\omega_2r_2$. Let the fluid surface be given by
$f(r_1,r_2)=r_2-h(r_1)=0$.

The stress tensor evaluated on such a fluid configuration is given
by
%
\begin{equation*}
\begin{split}
  T^{\mu\nu}_\mathrm{perfect} &=
    \begin{pmatrix}
      \gamma^2(\rho+(v_1^2+v_2^2)P) & 0 & \gamma^2\omega_1(\rho+P) & 0 & \gamma^2\omega_2(\rho+P) \\
      0 & P & 0 & 0 & 0 \\
      \gamma^2\omega_1(\rho+P) & 0 & \frac{\gamma^2}{r_1^2}(v_1^2\rho+(1-v_2^2)P) & 0 & \gamma^2\omega_1\omega_2(\rho+P) \\
      0 & 0 & 0 & P & 0 \\
      \gamma^2\omega_2(\rho+P) & 0 & \gamma^2\omega_1\omega_2(\rho+P) & 0 & \frac{\gamma^2}{r_2^2}(v_2^2\rho+(1-v_1^2)P) \\
    \end{pmatrix}
     \\
%\end{split}
%\end{equation*}
%%
%\begin{equation*}
%\begin{split}
  T^{\mu\nu}_\mathrm{dissipative} &=-\kappa\gamma^2
    \begin{pmatrix}
      0        & \p_{r_1}         & 0                & \p_{r_2}         & 0 \\
      \p_{r_1} & 0                & \omega_1\p_{r_1} & 0                & \omega_2\p_{r_1} \\
      0        & \omega_1\p_{r_1} & 0                & \omega_1\p_{r_2} & 0 \\
      \p_{r_2} & 0                & \omega_1\p_{r_2} & 0                & \omega_2\p_{r_2} \\
      0        & \omega_2\p_{r_1} & 0                & \omega_2\p_{r_2} & 0 \\
    \end{pmatrix}\frac{\tloc}{\gamma}
     \\
%\end{split}
%\end{equation*}
%%
%\begin{equation*}
%\begin{split}
  T^{\mu\nu}_\mathrm{surface} &=\frac{\sigma\delta(r_2-h(r_1))}{\sqrt{1+h'^2}}
    \begin{pmatrix}
      1+h'^2 & 0   & 0                     & 0     & 0 \\
      0      & -1  & 0                     & -h'   & 0 \\
      0      & 0   & -\frac{1+h'^2}{r_1^2} & 0     & 0 \\
      0      & -h' & 0                     & -h'^2 & 0 \\
      0      & 0   & 0                     & 0     & -\frac{1+h'^2}{r_2^2} \\
    \end{pmatrix}
\end{split}
\end{equation*}
%

As usual, we will temporarily ignore
$T^{\mu\nu}_\mathrm{dissipative}$, justifying this when we find that
$\tloc\propto\gamma$. The nontrivial equations of motion that follow
from \eqref{Epconsv:eq} take the form
%
\begin{equation}\label{5deom:eq}
\begin{split}
%  0 = \nabla_\mu T^{\mu t}
%   &= \p_{r_1}T^{r_1 t} +\p_{r_2}T^{r_2 t}
%      +\frac{1}{r_1}T^{r_1 t} +\frac{1}{r_2}T^{r_2 t}\\
%   &= 0\\
  0 = \nabla_\mu T^{\mu r_1}
%   &= \p_{r_1}T^{r_1 r_1} +\p_{r_2}T^{r_2 r_1}
%      +\frac{1}{r_1}T^{r_1 r_1} +\frac{1}{r_2}T^{r_2 r_1} -r_1T^{\phi_1 \phi_1}\\
   &= \p_{r_1}P -\gamma^2\omega_1^2r_1(\rho+P) -\sigma G h'(r_1)\delta(r_2-h(r_1))\\
%  0 = \nabla_\mu T^{\mu \phi_1}
%   &= \p_{r_1}T^{r_1 \phi_1} +\p_{r_2}T^{r_2 \phi_1}
%      +\frac{3}{r_1}T^{r_1 \phi_1} +\frac{1}{r_2}T^{r_2 \phi_1}\\
%   &= 0\\
  0 = \nabla_\mu T^{\mu r_2}
%   &= \p_{r_1}T^{r_1 r_2} +\p_{r_2}T^{r_2 r_2}
%      +\frac{1}{r_1}T^{r_1 r_2} +\frac{1}{r_2}T^{r_2 r_2} -r_2T^{\phi_2 \phi_2}\\
   &= \p_{r_2}P -\gamma^2\omega_2^2r_2(\rho+P) +\sigma G \delta(r_2-h(r_1))
%   \\
%  0 = \nabla_\mu T^{\mu \phi_2}
%   &= \p_{r_1}T^{r_1 \phi_2} +\p_{r_2}T^{r_2 \phi_2}
%      +\frac{1}{r_1}T^{r_1 \phi_2} +\frac{3}{r_2}T^{r_2 \phi_2}\\
%   &= 0
\end{split}
\end{equation}
%
where
%
\begin{equation*}
  G = -\frac{r_1hh''+(1+h'^2)(hh'-r_1)}{r_1h(1+h'^2)^{3/2}}
\end{equation*}
%

Using the equation of state \eqref{thermint:eq}, the equations of
motion in the fluid interior
%
\begin{equation*}
%\begin{split}
  \pdiff{\rho}{v_1} = 3(\rho-\rz) \frac{2v_1}{1-v_1^2-v_2^2}  \qquad
  \pdiff{\rho}{v_2} = 3(\rho-\rz) \frac{2v_2}{1-v_1^2-v_2^2}
%\end{split}
\end{equation*}
%
are easily solved and we find
%
\begin{equation*}
  (\rho-\rz)(1-v_1^2-v_2^2)^3 = 5K\rz \qquad
  P = \rz\prn{\frac{K}{(1-v_1^2-v_2^2)^3}-1}
  \qquad \tloc = \gamma\brk{\frac{K\rz}{\lambda}}^{1/6}.
\end{equation*}
%
Integrating the equations of motion \eqref{5deom:eq} across the
surface we obtain the condition (the upper sign should be used for
upper surfaces)
%
\begin{equation}\label{5dsurf:eq}
  \frac{K}{(1-\omega_1^2r_1^2-\omega_2^2h^2)^3}-1 =
     \mp\frac{\sigma}{\rz}
     \frac{r_1hh''+(1+h'^2)(hh'-r_1)}{r_1h(1+h'^2)^{3/2}}
\end{equation}
%
%Let $y=\omega_1h$, $x=\omega_1r_1$, $w=\frac{\omega_2}{\omega_1}$.
%%
%\begin{equation*}
%  \frac{K}{(1-x^2-w^2y^2)^3}-1 =
%     \mp\tw_1
%     \frac{xyy''+(1+y'^2)(yy'-x)}{xy(1+y'^2)^{3/2}}
%\end{equation*}
%%

Unfortunately we have not yet been able to solve this equation; we
postpone further analysis of \eqref{5dsurf:eq} to future work. In
the rest of this appendix we qualitatively describe possible types
of solutions to these equations, and their bulk dual horizon
topologies.

\begin{figure}
%
\begin{center}
  \input{5ball.tpx}\hspace{0.5cm}
  \input{5ring.tpx}\hspace{0.5cm}
  \input{5torus.tpx}
  \caption{Topologies of five dimensional solutions.}\label{5d:fig}
\end{center}
%
\end{figure}



In fig.\ref{5d:fig}, we have sketched some possible topologies for
these solutions. The first touches both the $r_1=0$ and $r_2=0$ axes
and we refer to this as a ball. The second type only touches one of
these axes and we refer to this as a ring. The third type touches
neither of the axes, as the plasma has the topology of a solid
three-torus we refer to this as a torus.

Each of these could be pinched near either axis, and there could be
hollow versions (though the considerations of \S\ref{sec:4dim} make
it seem unlikely that hollow configurations will actually be
solutions).




The horizon topology of the dual black object can be found by
fibering three circles over the shapes in fig.\ref{5d:fig}. One of
these circles degenerates at each axis (the angular coordinates
$\phi_1$ and $\phi_2$), and the other degenerates on the fluid
surface (the Scherk-Schwarz circle). The topologies generated are:
%listed in table \ref{topology5:tab}.
%
%
%
%\begin{table}
  \begin{center}
  \begin{tabular}{|l|l|l|}
    \hline
    % after \\: \hline or \cline{col1-col2} \cline{col3-col4} ...
    Horizon topology & Plasma topology & Object \\
    \hline
    $S^5$                     & $B^4$                     & Ball \\
    $S^4\times S^1$           & $B^3\times S^1$           & Ring \\
    $S^3\times S^1\times S^1$ & $B^2\times S^1\times S^1$ & Torus \\
    $S^3\times S^2$           & $B^1\times S^3$           & Hollow ball \\
    $S^2\times S^2\times S^1$ & $B^1\times S^2\times S^1$ & Hollow ring \\
    $S^2\times S^1\times S^1\times S^1$
                              & $B^1\times S^1\times S^1\times S^1$
                                                          & Hollow Torus \\
    $S^1\times S^1\times S^1\times S^1\times S^1$
                              & None                      & None \\
    \hline
  \end{tabular}
  \end{center}
%  \caption{Topologies of gravity and plasma solutions}\label{topology5:tab}
%\end{table}
