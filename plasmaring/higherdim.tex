% -*- TeX -*- -*- UK -*-

\section{Four dimensional plasmarings}\label{sec:4dim}

In the rest of this paper we turn to a consideration of localised
plasma configurations in certain massive 4 and 5 dimensional field
theories obtained by compactifying related 5 and 6 dimensional CFTs
on a Scherk-Schwarz circle. Although the field theories in question
are not gauge theories (e.g. the 5 dimensional massive theory could
be obtained by compactifying the (0,2) theory on the world volume of
an M5 brane on a Scherk-Schwarz circle) they undergo first order
`deconfining' transitions and the high temperature phase of these
theories admits a fluid dynamical description. The fluid
configurations we will construct are dual to localised black holes
and black rings in Scherk-Schwarz compactified AdS$_6$ and AdS$_7$
respectively.

In this section we study stationary solutions of fluid dynamics in 3+1
dimensional field theories. Our study will be less thorough than our 3 dimensional
analysis above; we find solutions analogous to those in 3 dimensions, but we postpone the
complete parameterisation and study of the thermodynamic properties
of these solutions to future work. In Appendix \ref{sec:5dim}
we have derived the equations relevant to stationary fluid flow in 5 dimensions, but
we leave the study of their solutions (and their higher dimensional counterparts)
to future work.



%\subsection{Thermodynamics}\label{sec:therm4d}
%
%The relevant equations are
%%
%\begin{equation}\label{therm4d:eq}
%  \begin{split}
%    P &= \frac{\rho-5\rz}{4} \,,\\
%    T &= \prn{\frac{\rho-\rz}{4\lambda}}^{1/5}.
%  \end{split}
%\end{equation}

\subsection{Stress tensor and equations of motion}\label{sec:eom4d}

In this section we set up the equations of motion of our fluid. We
proceed in direct imitation of our analysis of $d=3$ above. We use
the metric
%
\begin{equation}\label{metric4d:eq}
  \dr s^2 = -\dr t^2 + \dr r^2 + r^2 \dr \phi^2 + \dr z^2\,.
\end{equation}
%
This gives the same non-zero Christoffel symbols as before
\eqref{chrst:eq}. We choose the origin so that $r=0$ is the axis of
rotation and there is a reflection symmetry in the plane $z=0$.

%The equations of motion come from \eqref{Epconsv:eq} again. We will
%be looking at stationary, axially symmetric configurations, so $\p_t
%T = \p_\phi T = 0$. Using \eqref{chrst:eq}, \eqref{Epconsv:eq}
%becomes
%%
%\begin{equation}\label{Epconsv4d:eq}
%  \begin{split}
%    0 &= \nabla_\mu T^{\mu t}
%       = \p_r T^{rt} + \p_z T^{zt} + \frac{1}{r}T^{rt}, \\
%    0 &= \nabla_\mu T^{\mu r}
%       = \p_r T^{rr} + \p_z T^{zr} + \frac{1}{r}T^{rr} - r T^{\phi\phi}, \\
%    0 &= \nabla_\mu T^{\mu \phi}
%       = \p_r T^{r\phi} + \p_z T^{z\phi} + \frac{3}{r}T^{r\phi}, \\
%    0 &= \nabla_\mu T^{\mu z}
%       = \p_r T^{rz} + \p_z T^{zz} + \frac{1}{r}T^{rz}.
%  \end{split}
%\end{equation}
%%

For our configurations, $u^\mu = \gamma(1,0,\omega,0)$ with
$\gamma=\prn{1-\omega^2r^2}^{-1/2}$. We assume that the surface can
be described by $f(r,z) = z- h(r)$. In the interior of the
fluid,This leads to the stress tensor
%
\begin{equation}\label{blkstr4d:eq}
 T^{\mu\nu}_\mathrm{perfect} =
 \begin{pmatrix}
  \gamma^2(\rho+\omega^2r^2P) & 0 & \gamma^2\omega(\rho+P)                  & 0 \\
  0                           & P & 0                                       & 0 \\
  \gamma^2\omega(\rho+P)      & 0 & \frac{\gamma^2}{r^2}(\omega^2r^2\rho+P) & 0 \\
  0                           & 0 & 0                                       & P \\
 \end{pmatrix}
\end{equation}
%
and the surface stress tensor
%
\begin{equation}\label{srfstr4d:eq}
 T^{\mu\nu}_\mathrm{surface} =
   \frac{\sigma \delta(z- h(r))}{\sqrt{1+h'(r)^2}}
 \begin{pmatrix}
  1+h'(r)^2 & 0      & 0                      & 0        \\
  0         & -1     & 0                      & -h'(r)   \\
  0         & 0      & -\frac{1+h'(r)^2}{r^2} & 0        \\
  0         & -h'(r) & 0                      & -h'(r)^2 \\
 \end{pmatrix}
\end{equation}
%

Just as in $d=3$, the only potentially nonzero term in
$T_\mathrm{dissipative}^{\mu \nu}$ is proportional to
$\diff{}{r}\brk{\frac{\tloc}{\gamma}}$. As in $d=3$, it will turn
out that this quantity vanishes on our solutions, so we simply
proceed setting $T^{\mu \nu}_{dissipative}$ to zero.

%All the other tensors that appear in \S\S\ref{sec:visc} are the
%same, maybe with an extra row/column of zeroes, except for the
%projection tensor, $P^{\mu\nu}$, that has a 1 in the lower right
%corner. So there are no contributions to the stress tensor, provided
%that $\diff{}{r}\brk{\frac{\tloc}{\gamma}} = 0$.

The equations of motion, $\nabla_\mu T^{\mu\nu}=0$, reduce to
%
\begin{equation}\label{eom4d:eq}
 \begin{split}
   0 &= \pdiff{P}{r} -\frac{\omega^2r}{1-\omega^2r^2}\,(\rho+P)
              \mp 2\sigma H h'(r)\, \delta(z-h(r))\,,\\
   0 &= \pdiff{P}{z} \pm 2\sigma H\, \delta(z-h(r))\,,
 \end{split}
\end{equation}
%
where the upper sign refers to the upper ($z>0$) surface and
%
\begin{equation}\label{meancurv:eq}
  H = \mp\frac{rh''+ h'(1+h'^2)}{2r(1+h'^2)^{3/2}}
\end{equation}
%
is the mean curvature of the surface \cite{Weisstein-SurfRevo:99}.

\subsection{Solutions}\label{sec:sol4d}

Our analysis so far has been rather general; to proceed further we
use the equations of state \eqref{thermint:eq}. We define
dimensionless variables as before
%
\begin{equation}\label{newvars4d:eq}
    \tw = \frac{\sigma\omega}{\rz} \,,   \qquad
    v = \omega r  \,,  \qquad
    \g(v) = \omega h(r)\,.
\end{equation}

Using the equation of state \eqref{thermint:eq}, we can rewrite
\eqref{eom4d:eq} in the bulkinterior of the fluid as
%
\begin{equation}\label{eombulk4d:eq}
 \begin{split}
  \frac{1}{\rho-\rz}\diff{\rho}{v} &=
                        \frac{5v}{1-v^2}\,,\\
                        \implies&
  \prn{\rho(v)-\rz}\prn{1-v^2}^{5/2} = 4K\rz\,,
 \end{split}
\end{equation}
%
where $K$ is an integration constant. This means that the pressure
and temperature are
%
\begin{equation}\label{PT4d:eq}
    P = \rz \prn{\frac{K}{(1-v^2)^{5/2}}-1}\,, \qquad
    T = \gamma \prn{\frac{K\rz}{\lambda}}^{1/5}\,,
\end{equation}
%
(notice that this justifies our neglect of heat flow).

Integrating \eqref{eom4d:eq} across an outer surface gives
%
\begin{equation}\label{bcnd4d:eq}
%\begin{split}
  P = 2\sigma H \quad \text{or} \quad
  \frac{K}{(1-v^2)^{5/2}}-1  =
    -\tw \frac{v\g''+ \g'(1+\g'^2)}{v(1+\g'^2)^{3/2}}\,.
%\end{split}
\end{equation}
%
%We can determine the appropriate branch of the square root on the
%RHS of the equation above as follows: at the outermost radius, the
%mean curvature - and hence the pressure - is positive and
%$\g'=-\infty$, so we should choose the positive square root.

This can be integrated once to give
%Let
%%
%\begin{equation*}
%  y = \frac{v\g'}{\sqrt{1+\g'^2}}\,, \qquad x = \sqrt{1-v^2}\,.
%\end{equation*}
%%
%Then \eqref{bcnd4d:eq} becomes
%%
%\begin{equation}\label{bcyx4d:eq}
%\begin{split}
%  \diff{y}{x} &= \frac{K x^{-5/2} -1}{2\tw} \\ \implies
%  y &= -\frac{K}{3\tw(1-v^2)^{3/2}} + \frac{v^2}{2\tw} +
%  \frac{C}{\tw}\,,
%\end{split}
%\end{equation}
\begin{equation}\label{bcyx4d:eq}
  \frac{v\g'}{\sqrt{1+\g'^2}} =
    -\frac{K}{3\tw(1-v^2)^{3/2}} + \frac{v^2}{2\tw}
     + \frac{C}{\tw}\,,
\end{equation}
%
where $C$ is another integration constant.

If we introduce a parameter $l$ equal to the distance along the
surface, measured from $(v,\g)=(\vo,0)$, we have
$\diff{l}{v}=-\sqrt{1+\g'^2}$. Then \eqref{bcyx4d:eq} can be written
as
%
\begin{equation}\label{intrinsic:eq}
\begin{split}
  \diff{\g}{l} &= \frac{2K-3(v^2+2C)(1-v^2)^{3/2}}
                            {6\tw v(1-v^2)^{3/2}}\,,\\
  \diff{v}{l} &= -\frac{\sqrt{36\tw^2v^2(1-v^2)^3
                           - \brk{2K - 3(v^2+2C)(1-v^2)^{3/2}}^2}}
                     {6\tw v (1-v^2)^{3/2}}\,,\\
  \diff{\g}{v} &= -\frac{2K-3(v^2+2C)(1-v^2)^{3/2}}
                     {\sqrt{36\tw^2v^2(1-v^2)^3
                           - \brk{2K - 3(v^2+2C)(1-v^2)^{3/2}}^2}}\,.\\
\end{split}
\end{equation}
%
It follows that the outer surface of our plasma configuration is
given by
%
\begin{equation}\label{profile:eq}
  \g(v)= \int_{\vo}^v \!\dr x \left(  -\frac{2K-3(x^2+2C)(1-x^2)^{3/2}}
                     {\sqrt{36\tw^2x^2(1-x^2)^3
                           - \brk{2K - 3(x^2+2C)(1-x^2)^{3/2}}^2}}
                           \right)
\end{equation}
%
Of course this only makes sense provided
%
\begin{equation}\label{sense:eq}
  6\tw x(1-x^2)^{3/2} \geq \abs{2K - 3(x^2+2C)(1-x^2)^{3/2}}
  \quad \forall\; x \in (v, v_0).
\end{equation}
%
Note also the conditions $\rho>\rz \implies K>0$ and, of course,
$0<\vi<\vo$.


Inner boundaries to the plasma configuration (if they exist) obey
the equation $P=-2\sigma H$. The equivalent of \eqref{intrinsic:eq},
with a new integration constant $D$ replacing $C$ (the integration
constant $K$ is a property of the plasma, not the surfaces), is
%
\begin{equation}\label{innerbnd:eq}
\begin{split}
  \diff{\g}{l} &= -\frac{2K-3(v^2+2D)(1-v^2)^{3/2}}
                            {6\tw v(1-v^2)^{3/2}}\,,\\
  \diff{v}{l} &= -\frac{\sqrt{36\tw^2v^2(1-v^2)^3
                           - \brk{2K - 3(v^2+2D)(1-v^2)^{3/2}}^2}}
                     {6\tw v (1-v^2)^{3/2}}\,,\\
  \diff{\g}{v} &= \frac{2K-3(v^2+2D)(1-v^2)^{3/2}}
                     {\sqrt{36\tw^2v^2(1-v^2)^3
                           - \brk{2K - 3(v^2+2D)(1-v^2)^{3/2}}^2}}\,.\\
\end{split}
\end{equation}
%
The profiles of such boundaries may be obtained by integrating the
equation above.

%The shapes we study below will have `turning points'; points of
%local maxima and minima of $\g(v)$ at which $\g'(v)=0$ and points of
%local maxima and minima of $v(\g)$ (at which $\g'(v)$ diverges).
%Horizontal tangents that are points of inflection rather than
%extrema occur only if $\g''(v)=\g'(v)=0$. This highly non generic
%situation can occur only at special values of integration constants;
%we will ignore this possibility below. A similar statement is true
%of vertical tangents - generically these are turning points of our
%solutions.

Even before doing any analysis, we will find it useful to give names
to several easily visualised, topologically distinct fluid
configurations.


%
%\begin{description}
%  \item[Ordinary ball:] $\g'(v)$ is 0 at $v=0$, negative for
%  $0<v<\vo$ and $-\infty$ at $v=\vo$. $\g(v)$ is zero at $v=\vo$.
%  \item[Pinched ball:] $\g'(v)$ is 0 at $v=0$, positive for
%  $0<v<\vm$, 0 at $v=\vm$, negative for $\vm<v<\vo$ and $-\infty$ at
%  $v=\vo$. $\g$ is zero at $v=\vo$ and continuous at $v=\vm$.
%  \item[Ring:] $\g'(l)=v$  is $+\infty$ at $v=\vi$, is positive for
%  $\vi<v<\vm$, 0 at $v=\vm$, negative for $\vm<v<\vo$ and $-\infty$ at
%  $v=\vo$. $\g(v)$ is zero at $v=\vi,\vo$ and continuous at $v=\vm$.
%\end{description}
%
\begin{description}
  \item[Ordinary ball:] $v'(l)=\g(l)=0$ at $v=\vo$. $\g'(l)>0$ for
  $0<v<\vo$. $\g'(l)=0$ at $v=0$.
  \item[Pinched ball:] $v'(l)=\g(l)=0$ at $v=\vo$. $\g'(l)>0$ for
  $0<v<\vm$. $\g'(l)=0$ at $v=\vm$. $\g'(l)<0$ for $0<v<\vm$.
  $\g'(l)=0$ at $v=0$.
  \item[Ring:] $v'(l)=\g(l)=0$ at $v=\vo$. $\g'(l)>0$ for $\vm<v<\vo$.
  $\g'(l)=0$ at $v=\vm$. $\g'(l)<0$ for $\vi<v<\vm$. $v'(l)=\g(l)=0$
  at $v=\vi$, where $\vi<\vm<\vo$.
\end{description}
%
Examples of these surfaces can be seen in
figs.\ref{prof:fig}-\ref{surf:fig}. Each of these solutions could
have lumps of fluid eaten out of them. We will use the terms
%
\begin{description}
  \item[Hollow ball:] A ball (pinched or ordinary) with a ball cut out from its inside.
  \item[Hollow ring:] A ring with a ring cut out from its inside.
  \item[Toroidally hollowed ball:] A ball with a ring cut out from its inside.
\end{description}

It is easy to work out the horizon topology of the gravitational
solutions dual to the plasma topologies listed above.
\cite{Galloway:2005mf} have obtained a restriction on the topologies
of horizons of stationary black holes in any theory of gravity that
obeys the dominant energy condition; any product of spheres obeys
the conditions from their analysis.  Although the dominant energy
condition is violated in AdS space, in table \ref{topology:tab}, we
have listed all 4 dimensional horizons that are topologically
products of lower dimensional spheres, and note that all but one of
these configurations is obtained from the dual to plasma objects
named above ($B^3$ is a ball, $B^2$ is a disc and $B^1$ is an
interval). The last one, $T^4 = S^1\times S^1\times S^1\times S^1$,
is a marginal case of the theorem.

In the rest of this section we will determine all stationary,
rigidly spinning solutions of the equations of fluid dynamics
described above.

\begin{table}
  \begin{center}
  \begin{tabular}{|l|l|l|}
    \hline
    % after \\: \hline or \cline{col1-col2} \cline{col3-col4} ...
    Horizon topology & Plasma topology & Object \\
    \hline
    $S^4$                     & $B^3$                     & Ball \\
    $S^3\times S^1$           & $B^2\times S^1$           & Ring \\
    $S^2\times S^2$           & $B^1\times S^2$           & Hollow ball \\
    $S^2\times S^1\times S^1$ & $B^1\times S^1\times S^1$ & Hollow ring \\
    $S^1\times S^1\times S^1\times S^1$
                              & None                      & None \\
    \hline
  \end{tabular}
  \end{center}
  \caption{Topologies of gravity and plasma solutions}\label{topology:tab}
\end{table}



\subsubsection{Ordinary ball}\label{sec:oball4d}

We search for solutions of \eqref{profile:eq} for which $\g'(v)$
vanishes at $v=0$ and blows up at the outermost point of the surface
$v_0$; we also require that $\g$ decrease monotonically from $0$ to
$\vo$. The first condition sets $K=3C$.  The condition that $v'(l)$
is zero at $\vo$ may be used to determine $\tw$ as a function of
$\vo$ and $K$ from the linear equation
\begin{equation}\label{oballcnstr:eq}
  2K - (3\vo^2+2K)(1-\vo^2)^{3/2} = 6\tw\vo(1-\vo^2)^{3/2} \,,
\end{equation}
%
(the choice of positive square root comes from the fact that the LHS
above is positive).

%Near $v=0$ $\g'(v) \propto 3(K-1)v^2+\CO(v^4)$; thus our
%monotonicity condition requires $K>1$. It is not difficult to verify
%that these conditions are sufficient to guarantee our monotonicity
%requirement $\g'(v)>0$ in $(0,\vo)$. In order to see this
Note that the numerator of the formula for $\g'(v)$ be written as
%
\begin{equation*}
  2\brk{1-(1-v^2)^{3/2}}\prn{K-\frac{3v^2(1-v^2)^{3/2}}{2\brk{1-(1-v^2)^{3/2}}}}
\end{equation*}
%
and $2\prn{1-(1-v^2)^{3/2}}\geq3v^2(1-v^2)^{3/2}$. Thus, $K>1$
guarantees our monotonicity requirement. From \eqref{PT4d:eq}, we
see that this also ensures that the pressure is positive throughout
the ball.

In summary, the full set of ordinary ball solution is obtained by
substituting $C=K/3$ and $\omega= \omega(K, \vo)$ (obtained by
solving \eqref{oballcnstr:eq}) into \eqref{profile:eq}. This
procedure gives us a ball solution for any choice of $K>1$ and
$\vo>0$.

In figs.\ref{prof:fig},\ref{surf:fig} we present a plot of the profile $\g(v)$ for
the ball solution at $\vo=0.8$, $K=1.5$.


\subsubsection{Pinched ball}\label{sec:pball4d}

The pinched ball satisfies all the conditions of the ordinary ball
except for the monotonicity requirement on $\g(v)$; in fact the
function $\g(v)$ is required to first increase and then decrease as
$v$ runs from $0$ to $\vo$. It follows that $C$ and $\tw$ for these
solutions are determined as in the previous subsubsection ($C=K/3$
and $\omega $ from \eqref{oballcnstr:eq}) however the requirement
$\g''(v)>0$ at $v=0$ forces $K<1$. This ensures that $\g'(v)>0$ at
small $v$ and $\g'(v)<0$ at larger $v$. It also ensures that the
solution has negative pressure at the origin and positive pressure
at the outermost radius.

Not every choice of $(K,\vo)\in[0,1]$, however, yields an acceptable
pinched ball solution. As we decrease $\vo$ from 1, at fixed $K$, it
turns out that $\g(0)$ decreases, and in fact vanishes at a critical
value of $\vo$. Solutions at smaller $\vo$ are unphysical. The
physical domain,in $(K, \vo)$ space is given by the inequality
%
\begin{equation}\label{pballcnstr2:eq}
  \Delta \g \equiv -\!\!\int_0^{\vo} \diff{\g}{v}\,\dr v  =
  \int_0^{\vo}\!\!\!
      \frac{2K-(3v^2+2K)(1-v^2)^{3/2}}
           {\sqrt{36\tw^2v^2(1-v^2)^3-\brk{2K-(3v^2+2K)(1-v^2)^{3/2}}^2}}
      \,\dr v
  \geq 0\,.
\end{equation}
%
We should also ensure that \eqref{sense:eq} is not violated, i.e.
%
\begin{equation}\label{sensecnstr:eq}
  Q(\vo,K) \equiv \inf_{v \in (0,\vo)}\!
    \brc{36\tw^2 v^2(1-v^2)^{3} - \brk{2K - (3v^2+2K)(1-v^2)^{3/2}}^2}
    \geq 0\,.
\end{equation}
%
The boundary of the domain permitted by \eqref{pballcnstr2:eq}
is plotted in fig.\ref{pbreg:fig}. We have also plotted the boundary
of the region where \eqref{sensecnstr:eq} is violated. We see that
\eqref{pballcnstr2:eq} is the stricter constraint, and that
\eqref{sensecnstr:eq} is not violated for ordinary balls either. The
full set of pinched ball solutions is parameterised by values of
$\vo$ and $K$ in the region indiated in fig.\ref{pbreg:fig}.

\begin{figure}[tbh]
%
\begin{center}
  \input{pballreg.tpx}
  \caption{Domain of ball solutions.}\label{pbreg:fig}
\end{center}
%
\end{figure}


In figs.\ref{prof:fig},\ref{surf:fig} we present an example of the profile $\g(v)$
for the pinched ball solution at parameters $\vo=0.8$, $K=0.55$.


\subsubsection{Ring}\label{sec:ring4d}

The plasma of the ring configuration excludes the region $v<\vi$; as
this region omits $v=0$,  $K$ and $C$ are not constrained as before.

As $v'(l)$ vanishes at $\vi,\vo$ we have the following constraints
%
\begin{equation}\label{ringcnstr:eq}
\begin{split}
  2K - 3(\vi^2+2C)(1-\vi^2)^{3/2} &= -6\tw\vi(1-\vi^2)^{3/2} \,, \\
  2K - 3(\vo^2+2C)(1-\vo^2)^{3/2} &= 6\tw\vo(1-\vo^2)^{3/2} \,.
\end{split}
\end{equation}
%
the choice of negative/positive square roots comes from the
requirements that $\g'(l)<0$ at $v=\vi$ and $\g'(l)>0$ at $v=\vo$.
These equations may be used to solve for $C$ and $\tw$ as a function
of $K, \vi, \vo$. $K(\vo,\vi)$ may then be determined from the
requirement that $\g(\vi)=\g(\vo)=0$, i.e.
%
\begin{equation}\label{ringcnstr2:eq}
  \int_{\vi}^{\vo} \diff{\g}{v} \, \dr v =
  -\int_{\vi}^{\vo}\!\!\!\frac{2K-3(v^2+2C)(1-v^2)^{3/2}}
    {\sqrt{36\tw^2v^2(1-v^2)^3-\brk{2K - 3(v^2+2C)(1-v^2)^{3/2}}^2}}
  \, \dr v
  = 0\,.
\end{equation}
%
In practice, it is easier to first eliminate $K$ and $C$ using
\eqref{ringcnstr:eq}, then substitute $\vi=\tw\zi$, $\vo=\tw\ro$ and
use \eqref{ringcnstr2:eq} to solve for $\tw$ at fixed $\zi$ and
$\zo$. after this, one can determine $K$, $\vi$ and $\vo$ from
$\tw$, $\zi$ and $\zo$. We present a 3 dimensional plot of $K$ as a
function of $\vi$ and $\vo$ for $1<\zo<10$, $0.1<\zi/\zo<0.9$ in
fig.\ref{wplot:fig}. All of these solutions have $K>0$, as required.
Unfortunately, with this method, one cannot see if there is a
physically acceptable solution for the whole range of $0<\vi<\vo<1$.
It appears that there is a solution for every value of $\zi<\zo$.

\begin{figure}
%
\begin{center}
  \input{wplot.tpx}
  \caption{$K$ as a function of $\vi$ and $\vo$ for ring solutions.}\label{wplot:fig}
\end{center}
%
\end{figure}



In figs.\ref{prof:fig},\ref{surf:fig} we plot the profile function $\g(v)$ for the
ring solution at parameters
%$\vi=0.124$, $\vo=0.248$.
$\zi=10$, $\zo=20$.

%%
%\begin{equation*}
%  \frac{
%    3(1-v^2)^{3/2}
%    (
%      ((1-\vi^2)^{3/2}-(1-\vo^2)^{3/2})v^2
%      +\vi(2\tw-\vi) (1-\vi^2)^{3/2}
%      +\vo(2\tw+\vo) (1-\vo^2)^{3/2}
%    )
%    -3(\vi+\vo)(2\tw-\vi+\vo) (1-\vi^2)^{3/2} (1-\vo^2)^{3/2}
%   }
%   {\sqrt{
%    36 v^2 (1-v^2)^3 ((1-\vi^2)^{3/2}-(1-\vo^2)^{3/2})^2 \tw^2
%    -(
%      3(\vi+\vo)(2\tw-\vi+\vo) (1-\vi^2)^{3/2} (1-\vo^2)^{3/2}
%      -3(1-v^2)^{3/2} (
%                       ((1-\vi^2)^{3/2}-(1-\vo^2)^{3/2})v^2
%                       +\vi(2\tw-\vi) (1-\vi^2)^{3/2}
%                       +\vo(2\tw+\vo) (1-\vo^2)^{3/2}
%                      )
%      )^2
%   }}
%\end{equation*}
%%


\subsubsection{Hollow ball}\label{sec:hball}


In this subsection we will demonstrate the non-existence of rigidly
rotating hollow ball solutions to the equations of fluid dynamics.
Let us suppose such a solution did exist. The inner surface must
have vanishing gradient at $v=0$; this sets $D=K/3$. Now let the
outermost point of the eaten out region be $v=\vot$. The inner
surface must have a vertical tangent at $\vot$. This also implies
that the outer surface also has a vertical tangent at $\vot$ (the
condition for a vertical tangent is identical for an outer or inner
surface). However, such points saturate the inequality
\eqref{sense:eq} and, as discussed in \S\S\S\ref{sec:pball4d}, this
never happens in the interior of a ball. It follows that hollow ball
solutions do not exist.



\subsubsection{Hollow ring and toroidally hollowed ball}\label{sec:hring}

Let us first consider the possibility of the existence of a toridally hollowed ball
solution. Let the innermost and outermost part of the hollowed out region occur at
$v=\vit$ and  $v=\vot$ respectively. Let us define $a(v)= 6 \tw v(1-v^2)^{{3/2}}$
and $b(v)=-2K+3(v^2+2D)(1-v^2)^{{3/2}}$ where $D$ is the integration constant for
the hollow. From \eqref{innerbnd:eq} it must be that
%
\begin{equation*}
  a(\vot)=b(\vot)\qquad
  a(\vit)=-b(\vit)\qquad
  |b(v)|<|a(v)|\; \forall   v \in (\vit, \vot)
\end{equation*}
%
For these conditions to apply, $b(v)$ must start out negative at $v=\vit$, increase, turn positive, and cut the $a(v)$ curve from below at $v=\vot$. We have performed a rough numerical scan of allowed values of parameters $(K, \tw, D)$; it appears that this behaviour never occurs 
(although we do not, however, have a rigorous proof for this claim). For all physically 
acceptable values of parameters, the curve $b(v)$ appears to either stay entirely below 
$a(v)$ or to cut it from above. \footnote{We emphasise that this behaviour appears to be true only for $\tw>\tw_{min}(K)$ where $\tw_{min}(K)$ is the smallest allowed value of $\tw$ at fixed
$K$ (see fig.\ref{pbreg:fig}).  It is easy to arrange for $b(v)$ to cut $a(v)$ from below 
when $\tw$ is taken to be arbitrarily small at fixed $K$ and $D$, but this is unphysical.}

These considerations, which could presumably be converted into a proof with enough effort, 
lead us to believe that the existence of hollow balls is highly unlikely.
We believe that similar reasoning is likely to rule out the existence of hollow rings, although this is more difficult to explicitly verify, as our understanding of the parameter ranges for
acceptable ring solutions is incomplete.

In order to understand intuitively why hollow rings and toroidally hollow balls are unlikely,
note that the pressure at the inner and outermost parts of the hollowed out region is given by
%
\begin{equation*}
  P(\vit) = \rz\tw\prn{-|v''_{v=\vit}|+\frac{1}{\vit}}, \qquad
  P(\vot) = \rz\tw\prn{-|v''_{v=\vot}|-\frac{1}{\vot}},
\end{equation*}
%
where $v''_{v=\vit}$ is positive and $v''_{v=\vot}$ is negative.
Provided that $|v''_{v=\vit}|$ and $|v''_{v=\vot}|$ are  not
drastically different, we would require $P(\vit)>P(\vot)$. However,
the pressure increases monotonically with radius.

%Note that the innermost part of a ring surface has negative mean
%curvature whereas the outermost part has positive mean curvature.
%This means that they cannot form inner surfaces, as the pressure
%would have to be positive at the innermost part and negative at the
%outermost part, but pressure increases monotonically with radius.
%Thus, hollow rings are impossible.


In conclusion, we strongly suspect, but have not yet fully proved,
that the full set of rigidly rotating solutions to the equations of fluid dynamics
in $d=4$ is exhausted by ordinary balls, pinched balls and rings.


\begin{figure}
%
\begin{center}
  \input{obp.tpx}
  \input{pbp.tpx}
  \input{rp.tpx}
  \caption{Profile of the surface of an ordinary
ball, pinched ball and ring.}\label{prof:fig}
\end{center}
%
\end{figure}

\begin{figure}
%
\begin{center}
  \input{obs.tpx}
  \input{pbs.tpx}
  \input{rs.tpx}
  \caption{3D plot of the surface of an ordinary
ball, pinched ball and ring.}\label{surf:fig}
\end{center}
%
\end{figure}




%\begin{figure}[p]
%%
%\begin{center}
%  \input{obp.tpx}
%  \qquad
%  \input{obs.tpx}
%  \caption{Profile and 3D plot of the surface of an ordinary
%ball.}\label{obps:fig}
%\end{center}
%%
%\end{figure}
%
%
%
%\begin{figure}[p]
%%
%\begin{center}
%  \input{pbp.tpx}
%  \qquad
%  \input{pbs.tpx}
%  \caption{Profile and 3D plot of the surface of a pinched
%ball.}\label{pbps:fig}
%\end{center}
%%
%\end{figure}
%
%
%\begin{figure}[p]
%%
%\begin{center}
%  \input{rp.tpx}
%  \qquad
%  \input{rs.tpx}
%  \caption{Profile and 3D plot of the surface of a
%ring.}\label{rps:fig}
%\end{center}
%%
%\end{figure}

%\clearpage
